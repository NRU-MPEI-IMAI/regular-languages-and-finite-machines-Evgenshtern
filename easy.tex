\documentclass{article}
\usepackage[T2A]{fontenc}
\usepackage[utf8]{inputenc}
\usepackage[russian]{babel}
\usepackage{textcomp}
\usepackage{color}
\usepackage{xspace}
\usepackage{multirow}
\usepackage{amsmath,amsfonts,amsthm,amssymb,amsbsy,amstext,amscd,amsxtra,multicol}
\usepackage{indentfirst}
\usepackage{verbatim}
\usepackage[left=2cm, right=2cm, top=2cm, bottom=2cm, bindingoffset=0cm]{geometry}
\usepackage[pdf]{graphviz}
\usepackage{morewrites}
\usepackage[notransparent]{svg}     % svg




\begin{document}
    \vbox{%
        \hfill%
        \vbox{%
            \hbox{\large Выполнил студент группы А-13а-19 \par}%
            \hbox{\large Легкий Евгений Андреевич \par}%
            \hbox{\large 8 апреля 2022 г.\par}%
        }%
    } 
    
    \begin{center}
        {\LARGE  \textbf{Домашняя работа №1}\par}
        {\Large по Теоретическим моделям вычислений\par}
    \end{center}
    
    \section*{Задание 1}
     Построить конечные автоматы
    \begin{enumerate}
        \item \(L_1=\{\omega\in\{a,b,c\}^* : |\omega|_c = 1 \} \)
        \begin{center}
            \begin{figure}[htbp]
                \centering
                \includesvg{1.1.dot.svg}
            \end{figure}
        \end{center}

     \item \(L_2=\{\omega\in\{a,b\}^* : |\omega|_a \leqslant 2, |\omega|_b \geqslant 2 \} \)
     
     Рассмотрим автоматы 
     \(A=\{\omega\in\{a,b\}^* : |\omega|_a \leqslant 2 \} \) 
     и 
     \(B=\{\omega\in\{a,b\}^* : |\omega|_B \geqslant 2 \} \)
        \begin{center}
            \begin{figure}[htbp]
                \centering
                \includesvg[scale=0.7]{1.2.a.dot.svg}
            \end{figure}
        \end{center}
    и 
    \(B=\{\omega\in\{a,b\}^* : |\omega|_B \geqslant 2 \} \)
        \begin{center}
            \begin{figure}[htbp]
                \centering
                \includesvg[scale=0.7]{1.2.b.dot.svg}
            \end{figure}
        \end{center}
    Так как \(L_2 = A \cup B = A \times B \)
    \begin{center}
        \begin{tabular} {|c|c|c|}
            \hline
             & a & b \\
            \hline
            \((q_1,q_4)\) & \((q_2,q_4)\) & (\(q_1,q_5)\) \\
            \hline
            \((q_1,q_5)\) & \((q_2,q_5)\) & \((q_1,q_6)\) \\
            \hline
            \((q_1,q_6)\) & \((q_2,q_6)\) & \((q_1,q_6)\) \\
            \hline
            \((q_2,q_4)\) & \((q_3,q_4)\) & \((q_2,q_5)\) \\
            \hline
            \((q_2,q_5)\) & \((q_3,q_5)\) & \((q_2,q_6)\) \\
            \hline
            \((q_2,q_6)\) & \((q_3,q_6)\) & (\(q_2,q_6)\) \\
            \hline
            \((q_3,q_4)\) & $(\emptyset)$ & \((q_3,q_5)\) \\
            \hline
            \((q_3,q_5)\) & $(\emptyset)$ & \((q_3,q_6)\) \\
            \hline
            \((q_3,q_6)\) & $(\emptyset)$ & \((q_3,q_6)\) \\
            \hline
        \end{tabular}
    \end{center}
    И окончательный ответ
    \begin{center}
            \begin{figure}[htbp]
                \centering
                \includesvg[scale=0.7]{1.2.dot.svg}
            \end{figure}
        \end{center}
    \item \(L_3=\{\omega\in\{a,b\}^*:|\omega|_a \neq |\omega|_b \}\) \\
    Для описания языка необходо запоминать количество символов одного типа, что нельзя сделать с помощью ДКА
    \item \(L_4=\{\omega\in\{a,b\}^* : \omega \omega = \omega \omega \omega \} \) \\
    Язык описывает только пустые слова
    \begin{center}
        \begin{figure}[htbp]
            \centering
            \includesvg[scale=0.7]{1.4.dot.svg}
        \end{figure}
    \end{center}
    \end{enumerate}
    \section*{Задание 2}
    Построить конечные автоматы, распознающие слудеющие языки, используя прямое произведение:
    \begin{enumerate}
        \item \(L_1=\{\omega\in\{a,b\}^* : |\omega|_a \geqslant 2 \wedge |\omega|_b
        \geqslant 2 \} \) \\
        
    Рассмотрим автоматы 
    \(A=\{\omega\in\{a,b\}^* : |\omega|_a \geqslant 2 \} \) 
     \begin{center}
            \begin{figure}[htbp]
                \centering
                \includesvg[scale=0.7]{2.1.a.dot.svg} 
            \end{figure}
        \end{center}
    
    \newpage
    
    и \(B=\{\omega\in\{a,b\}^* : |\omega|_B \geqslant 2 \} \) 
     \begin{center}
            \begin{figure}[htbp]
                \centering
                \includesvg[scale=0.7]{2.1.a.dot.svg}
            \end{figure}
        \end{center}
        
        
    \(L_1 = A \cup B = A \times B \)  
    
    Построим таблицу переходов
    \begin{center}
        \begin{tabular} {|c|c|c|}
            \hline
             & a & b \\
            \hline
            \((q_1,q_4)\) & \((q_2,q_4)\) & (\(q_1,q_5)\) \\
            \hline
            \((q_1,q_5)\) & \((q_2,q_5)\) & \((q_1,q_6)\) \\
            \hline
            \((q_1,q_6)\) & \((q_2,q_6)\) & \((q_1,q_6)\) \\
            \hline
            \((q_2,q_4)\) & \((q_3,q_4)\) & \((q_2,q_5)\) \\
            \hline
            \((q_2,q_5)\) & \((q_3,q_5)\) & \((q_2,q_6)\) \\
            \hline
            \((q_2,q_6)\) & \((q_3,q_6)\) & (\(q_2,q_6)\) \\
            \hline
            \((q_3,q_4)\) & \((q_3,q_4)\) & \((q_3,q_5)\) \\
            \hline
            \((q_3,q_5)\) &  \((q_3,q_5)\) & \((q_3,q_6)\) \\
            \hline
            \((q_3,q_6)\) &  \((q_3,q_6)\) & \((q_3,q_6)\) \\
            \hline
        \end{tabular}
    \end{center}
    Финальный результат
    \begin{center}
            \begin{figure}[htbp]
                \centering
                \includesvg[scale=0.7]{2.1.dot.svg}
            \end{figure}
        \end{center}
    \item \(L_2=\{\omega \in\{a,b\}^* : |\omega| \geqslant 3 \wedge |\omega| \text{ нечётное} \} \) \\
    \newpage 
     Рассмотрим автоматы 
     \(A=\{\omega\in\{a,b\}^* : |\omega| \geqslant 3 \} \)
     \begin{center}
            \begin{figure}[htbp]
                \centering
                \includesvg[scale=0.7]{2.2.a.dot.svg}
            \end{figure}
        \end{center}
        
     и 
     \(B=\{\omega\in\{a,b\}^* : |\omega| \text{ нечётное} \} \)
    \begin{center}
            \begin{figure}[htbp]
                \centering
                \includesvg[scale=0.7]{2.2.b.dot.svg}
            \end{figure}
        \end{center}
    \(L_2 = A \cup B = A \times B \)    \\
    Построим таблицу переходов
    \begin{center}
        \begin{tabular} {|c|c|c|}
            \hline
             & a & b \\
            \hline
            \((q_1,q_5)\) & \((q_2,q_6)\) & (\(q_2,q_6)\) \\
            \hline
            \((q_1,q_6)\) & \((q_2,q_5)\) & \((q_1,q_5)\) \\
            \hline
            \((q_2,q_5)\) & \((q_3,q_6)\) & \((q_3,q_6)\) \\
            \hline
            \((q_2,q_6)\) & \((q_3,q_5)\) & \((q_3,q_5)\) \\
            \hline
            \((q_3,q_5)\) & \((q_4,q_6)\) & \((q_4,q_6)\) \\
            \hline
            \((q_3,q_6)\) & \((q_4,q_5)\) & (\(q_4,q_5)\) \\
            \hline
            \((q_4,q_5)\) & \((q_4,q_6)\) & \((q_4,q_6)\) \\
            \hline
            \((q_4,q_6)\) &  \((q_4,q_5)\) & \((q_4,q_5)\) \\
            \hline
        \end{tabular}
    \end{center}
    \begin{center}
            \begin{figure}[htbp]
                \centering
                \includesvg[scale=0.5]{2.2_pred_dot.svg}
            \end{figure}
        \end{center}
    \newpage 
    Так как узлы \((q_1,q_6)\), \((q_2,q_5)\) и \((q_3,q_6)\) недостижимы, то их можно убрать
    \begin{center}
            \begin{figure}[htbp]
                \centering
                \includesvg[scale=0.5]{2.2_post.dot.svg}
            \end{figure}
        \end{center}
    \item \(L_3=\{\omega \in\{a,b\}^* : |\omega|_a \text{ чётно} \wedge |\omega|_b \text{ кратно } 3 \} \) \\   
     Рассмотрим автоматы 
     \(A=\{\omega\in\{a,b\}^* : |\omega|_a \text{ чётно} \} \) 
      \begin{center}
            \begin{figure}[htbp]
                \centering
                \includesvg[scale=0.5]{2.3.a.dot.svg}
            \end{figure}
        \end{center}
     и 
     \(B=\{\omega\in\{a,b\}^* : |\omega|_B \text{ кратно } 3 \} \)    
     \begin{center}
            \begin{figure}[htbp]
                \centering
                \includesvg[scale=0.5]{2.3.b.dot.svg}
            \end{figure}
        \end{center}
     \(L_3 = A \cup B = A \times B \)
    \begin{center}
        \begin{tabular} {|c|c|c|}
            \hline
             & a & b \\
            \hline
            \((q_1,q_3)\) & \((q_2,q_3)\) & (\(q_1,q_4)\) \\
            \hline
            \((q_1,q_4)\) &  $(\emptyset)$ & \((q_1,q_5)\) \\
            \hline
            \((q_1,q_5)\) &  $(\emptyset)$ & \((q_1,q_6)\) \\
            \hline
            \((q_1,q_6)\) & \((q_2,q_3)\) & $(\emptyset)$ \\
            \hline
            \((q_2,q_3)\) & \((q_1,q_3)\) & \((q_2,q_4)\) \\
            \hline
            \((q_2,q_4)\) &  $(\emptyset)$ & (\(q_2,q_5)\) \\
            \hline
            \((q_2,q_5)\) &  $(\emptyset)$ & \((q_2,q_6)\) \\
            \hline
            \((q_2,q_6)\) &  \((q_1,q_3)\) & $(\emptyset)$ \\
            \hline
        \end{tabular}
    \end{center}
    \newpage
    
    Финальный результат
    \begin{center}
            \begin{figure}[htbp]
                \centering
                \includesvg[scale=0.6]{2.3.dot.svg}
            \end{figure}
        \end{center}
        
    \item \(L_4= \neg L_3\) \\
    Так как \(T_4 = Q_3 \setminus T_3 \) \\
    \(T_4 = {(q_1,q_3), (q_1,q_4), (q_2,q_3), (q_1,q_5), (q_2,q_4), (q_2,q_5)}\)
    \begin{center}
            \begin{figure}[htbp]
                \centering
                \includesvg[scale=0.6]{2.4.dot.svg}
            \end{figure}
        \end{center}
    \item \(L_5= L_2 \setminus L_3\) \\
    \(L_5 = L_2 \cup \neg L_3 = L_2 \times \neg L_3 = L_2 \times \neg L_3 = L_2 \times L_4 \) 
    
     \begin{center}
            \begin{figure}[htbp]
                \centering
                \includesvg[scale=0.6]{2.5.a.dot.svg}
            \end{figure}
        \end{center}
       \newpage    
     \begin{center}
            \begin{figure}[htbp]
                \centering
                \includesvg[scale=0.6]{2.5.b.dot.svg}
            \end{figure}
        \end{center}
    Финальный результат  
     \begin{center}
            \begin{figure}[htbp]
                \centering
                \includesvg[scale=0.4]{2.5.dot.svg}
            \end{figure}
        \end{center}
    \end{enumerate}
    
    \section*{Задание 3}
    Построить минимальные ДКА по регулярным выражениям:
    \begin{enumerate}
    \item \((ab + aba)^{*}a\) \\
    Строим НКА по регулярному выражению
    \begin{center}
            \begin{figure}[htbp]
                \centering
                \includesvg[scale=0.6]{3.1_lamda.dot.svg}
            \end{figure}
        \end{center}
    \newpage
    Преобразуем в ДКА 
     \begin{center}
            \begin{figure}[htbp]
                \centering
                \includesvg[scale=0.6]{3.1_NSA.dot.svg}
            \end{figure}
        \end{center}
    Построим минимальный ДКА по алгоритму Томсона
     \begin{center}
             \begin{tabular}{ |c|c|c| } 
                \hline
                \(Q\) & \(a\) & \(b\) \\
                \hline\hline
                q1 & q2q3q4 & - \\
                \hline
                q2q3q4 & - & q1q5 \\
                \hline
                q1q5 & q1q2q3q4 & - \\
                \hline
                q1q2q3q4 & q2q3q4 & q1q5 \\
                \hline
            \end{tabular}
        \end{center}
    Ответ
    \begin{center}
            \begin{figure}[htbp]
                \centering
                \includesvg[scale=0.6]{3.1_FSA.dot.svg}
            \end{figure}
        \end{center}
        
    \item \(a(a(ab)^{*}b)^{*}(ab)^{*}\)\\
    Строим НКА по регулярному выражению
    \begin{center}
            \begin{figure}[htbp]
                \centering
                \includesvg[scale=0.6]{3.2_NSA.dot.svg}
            \end{figure}
        \end{center}
    \newpage
    Преобразуем в ДКА 
     \begin{center}
            \begin{figure}[htbp]
                \centering
                \includesvg[scale=0.6]{3.2_FSA.dot.svg}
            \end{figure}
        \end{center}
    А теперь минимализируем      
    \begin{center}
            \begin{figure}[htbp]
                \centering
                \includesvg[scale=0.6]{3.2_FSA_min.dot.svg}
            \end{figure}
        \end{center}    
    \item \((a + (a + b)(a + b)b)^*\)\\
    Строим НКА по регулярному выражению
    \begin{center}
            \begin{figure}[htbp]
                \centering
                \includesvg[scale=0.4]{3.3_NSA.dot.svg}
            \end{figure}
        \end{center}
    \newpage
    Преобразуем в ДКА 
     \begin{center}
            \begin{figure}[htbp]
                \centering
                \includesvg[scale=0.6]{3.3_FSA.dot.svg}
            \end{figure}
        \end{center}
    А теперь минимализируем      
    \begin{center}
            \begin{figure}[htbp]
                \centering
                \includesvg[scale=0.6]{3.3_FSA_min.dot.svg}
            \end{figure}
        \end{center}
    \newpage
    \item \((b+c)((ab)^*c + (ba)^*)^*\)\\
    Строим НКА по регулярному выражению
    \begin{center}
            \begin{figure}[htbp]
                \centering
                \includesvg[scale=0.4]{3.4_NSA.dot.svg}
            \end{figure}
        \end{center}
    Преобразуем в ДКА 
     \begin{center}
            \begin{figure}[htbp]
                \centering
                \includesvg[scale=0.6]{3.4_FSA.dot.svg}
            \end{figure}
        \end{center}
    \newpage
    А теперь минимализируем      
    \begin{center}
            \begin{figure}[htbp]
                \centering
                \includesvg[scale=0.6]{3.4_FSA_min.dot.svg}
            \end{figure}
        \end{center}    
    
    \end{enumerate}
    \section*{Задание 4}
    Определить является ли язык регулярным или нет:
    \begin{enumerate}
     \item   \(L_1 = \{(aab)^nb(aba)^m \mid n \geq 0 , m \geq 0\}\) \\
      \begin{center}
            \begin{figure}[htbp]
                \centering
                \includesvg[scale=0.6]{4.1_FSA.dot.svg}
            \end{figure}
        \end{center}  

    Он регулярный, так как можно построить ДКА
    //
  \item \(L = \{uaav : u \in \{a, b\}^*, \; v \in \{a, b\}^*, |u|_b \geqslant |v|_a\}\)\\
        Применим лемму о разрастании. Зафиксируем \(\forall n \in \mathbb{N} \) и рассмотрим слово \(\omega = b^{n}aaa^{n}, \; |\omega| = 2n + 2 \geq n\). Теперь рассмотрим все разбиения этого слова \(\omega = xyz\) такие, что \(|y| \neq 0, \; |xy| \leq n\):
        $$x = b^{k}, \; y = b^{l}, \; z = b^{n - k - l}aaa^n,$$ \begin{center}
            где \(1 \leq k + l \leq n \; \wedge \; l > 0\)
        \end{center} 
        Дргуих разбиенний, удовлетворяющих данным условиям, нет.
        Для любого из таких разбиений слово \(xy^0z \notin L\). Лемма не выполняется, значит, \(L\) не регулярный язык.
        
        \item \(L = \{a^mw : w \in \{a, b\}^{*}, \; 1 \geqslant |w|_b \geqslant m\}\)\\
        Применим лемму о разрастании. Зафиксируем \(\forall n \in \mathbb{N} \) и рассмотрим слово \(\omega = a^nb^n, \; |\omega| = 2n \geqslant n\). Теперь рассмотрим все разбиения этого слова \(\omega = xyz\) такие, что \(|y| \neq 0, \; |xy| \leq n\):
        $$x = a^{l}, \; y = a^{m}, \; z = a^{n-l-m}b^{n},$$ 
        \begin{center}
            где \(l + k \leqslant n \; \wedge \; m \ne 0\)
        \end{center} 
        Дргуих разбиенний, удовлетворяющих данным условиям, нет. Теперь выполним накачку: 
        $$xy^{i}z = a^{l}(a^{m})^{i}a^{n-l-m}b^{n} = a^{n-mi}b^{n} \notin L, \; i 
        \geqslant 0 \in \mathbb{N} $$
        Лемма не выполняется, значит, \(L\) не регулярный язык.
        
        \item \(L = \{a^{k}b^{m}a^{n} : k = n \vee m > 0\}\)\\
        Применим лемму о разрастании. Зафиксируем \(\forall n \in \mathbb{N} \) и рассмотрим слово \(\omega = a^nba^n, \; |\omega| = 2n + 1 \geqslant n\). Теперь рассмотрим все разбиения этого слова \(\omega = xyz\) такие, что \(|y| \neq 0, \; |xy| \leq n\):
        $$x = a^{k}, \; y = a^{m}, \; z = a^{n-k-m}ba^{n},$$ 
        \begin{center}
            где \(k + m \leqslant n \; \wedge \; m \ne 0\)
        \end{center} 
        Другуих разбиенний, удовлетворяющих данным условиям, нет.
        Теперь выполним накачку: 
        $$xy^{i}z = a^{k}(a^{m})^{i}a^{n-k-m}ba^{n} = a^{n+m(i-1)}ba^{n} \notin L, \; i 
        \geqslant 2 \in \mathbb{N} $$
        Получили противоречие, лемма не выполняется, значит, \(L\) не регулярный язык.
        
        \item \(L = \{ucv : u \in \{a, b\}^*, \; v \in \{a, b\}^*, u \ne v^R \}\)\\
        Применим лемму о разрастании. Зафиксируем \(\forall n \in \mathbb{N} \) и рассмотрим слово \(\omega = (ab)^nc(ab)^n = \alpha_1\alpha_2...\alpha_{4n+1}, \; |\omega| = 4n + 1 \geqslant n\). Теперь рассмотрим все разбиения этого слова \(\omega = xyz\) такие, что \(|y| \neq 0, \; |xy| \leq n\):
        $$x = \alpha_1\alpha_2...\alpha_k, \; y = \alpha_{k+1}...\alpha_{k+m}, \; z = \alpha_{k+m+1}...\alpha_{4n+1}c(ab)^n,$$
        \begin{center}
            где \(k + m \leqslant n \; \wedge \; m \ne 0\)
        \end{center} 
        Дргуих разбиенний, удовлетворяющих данным условиям, нет. Теперь выполним накачку: 
        $$xy^{i}z = (\alpha_1\alpha_2...\alpha_k)(\alpha_{k+1}...\alpha_{k+m})^i(\alpha_{k+m+1}...\alpha_{4n+1}c(ab)^n)$$
        При \(i = 2\) имеем:
        $$xy^{2}z = (\alpha_1\alpha_2...\alpha_k)(\alpha_{k+1}...\alpha_{k+m})^2(\alpha_{k+m+1}...\alpha_{4n+1}c(ab)^n) \notin L$$
        Лемма не выполняется, значит, \(L\) не регулярный язык.
    \end{enumerate}
\end{document}
    
    
